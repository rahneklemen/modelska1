\documentclass[slovene,11pt,a4paper]{article}
%\usepackage{fullpage}

%dodatni paketki:
\usepackage{graphicx}
\usepackage{amsmath,amsfonts,amsthm} %matematicni paket
\usepackage{color} % omogoča barvno pisanje
\usepackage[utf8]
{inputenc}
\usepackage[slovene]{babel} % slovenski jezik/hyphenation
\usepackage{hyperref} %naredi vse povezave rečerenc, kazala,...
\numberwithin{equation}{section} % Number equations within sections (i.e. 1.1, 1.2, 2.1, 2.2 instead of 1, 2, 3, 4)
\numberwithin{figure}{section} % Number figures within sections (i.e. 1.1, 1.2, 2.1, 2.2 instead of 1, 2, 3, 4)
\numberwithin{table}{section} % Number tables within sections (i.e. 1.1, 1.2, 2.1, 2.2 instead of 1, 2, 3, 4)
\usepackage{eurosym} %za znak €




\usepackage[margin=2cm]{geometry}
\include{template}




\begin{document}
\begin{titlepage}

\newcommand{\HRule}{\rule{\linewidth}{0.5mm}} % Defines a new command for the horizontal lines, change thickness here

\center % Center everything on the page

%----------------------------------------------------------------------------------------
%	LOGO
%----------------------------------------------------------------------------------------

%\includegraphics{Logo}\\[1cm] % Include a department/university logo - this will require the graphicx package
 
%----------------------------------------------------------------------------------------

\includegraphics[width=2cm]{slike/aaa}\\[0.5cm]
 
%----------------------------------------------------------------------------------------
%	NASLOV DELA
%----------------------------------------------------------------------------------------
\textit{Univerza v Ljubljani}\\
\textit{Fakulteta za {\color{red}matematiko in fiziko}}\\[0.5cm]

\emph{Oddelek za fiziko}\\[0.5cm] % Oddelek za fiziko


%----------------------------------------------------------------------------------------
%	TITLE SECTION
%--------------------------------------------------------------------------------------
\HRule \\[0.4cm]
\huge {\bfseries 2. naloga: Linearno programiranje}\\[0.4cm] % NASLOV SEMINARJA
\HRule \\[0.5cm] 

 \textsc{\large Poročilo pri predmetu modelska analiza 1}\\
 \textsc{\large 2015/2016}\\[1cm] % SEMINASKO DELO
 
%----------------------------------------------------------------------------------------
%	AUTHOR SECTION
%----------------------------------------------------------------------------------------



% If you don't want a supervisor, uncomment the two lines below and remove the section above
\Large \emph{Avtor:}\\
Klemen \textsc{Rahne}\\
28152028\\[2cm]
%----------------------------------------------------------------------------------------
%	DATUM
%----------------------------------------------------------------------------------------

{\large \today } \\[0.5cm] % Date, change the \today to a set date if you want to be precise

	

\end{titlepage}
%
%\maketitle
%
%\thispagestyle{empty}
%\vspace{-1.5cm}
%
% Med tipične primere, ki jih lahko učinkovito rešimo z metodami
% linearnega programiranja, sodi sestavljanje diet za hujšanje,
% zdravljenje ali športne aktivnosti. Za dani nabor živil določamo
% njihove količine, pri čemer moramo zadostiti različnim omejitvam. Med
% drugim moramo zagotoviti priporočene dnevne odmerke mineralov,
% vitaminov in hranilnih snovi, omejiti pri vnos maščob, ogljikovih
% hidratov ter telesu škodljivih snovi, hkrati pa zagotoviti, da
% energijska vrednost ustreza zahtevam posameznika.
%
% Vnos vsake izmed hranilnih snovi je linearna funkcija količin živil
% in je natanko določena z njihovo sestavo. Od vrste diete pa je
% odvisno, katere parametre omejimo in katere minimiziramo.
%
%\begin{enumerate}
%\item
%
%  Tabela v datoteki {\tt
%    http://predmeti.fmf.uni-lj.si/modelska/podatki/tabela-zivil.dat}
%  vsebuje podatke o energijski vrednosti ter vsebnosti maščob,
%  ogljikovih hidratov, proteinov, kalcija in železa v nekaj živilih.
%  Minimiziraj število kalorij, če je priporočen minimalni dnevni vnos
%  $70\,\rm g$ maščob, $310\,\rm g$ ogljikovih hidratov, $50\,\rm g$
%  proteinov, $1000\,\rm mg$ kalcija ter $18\,\rm mg$ železa. Upoštevaj
%  tudi, da naj dnevni obroki količinsko ne presežejo dveh kilogramov
%  hrane.
%
%\item
%  Kako se rezultat razlikuje, če zahtevamo minimalno $2000\,\rm
%  kcal$ in namesto energije minimiziramo vnos maščob?
%
%\item
%  Ker rešujemo poenostavljen problem s samo $5$ parametri na živilo,
%  so lahko rezultati nerealistični. Lahko z omejitvijo količine
%  posameznih živil v obroku izboljšaš uravnovešenost prehrane?
%
%
%\item[*] {\sl (neobvezno)} Poišči cene živil in poskusi namesto
%  kalorij minimizirati ceno. Kako se varčevanje odraža na zdravi prehrani?
%
%\end{enumerate}
%
%\hrule
%
%{\small
%\paragraph{\small Linearno programiranje:}
%Linearni optimizacijski problem formuliramo kot optimizacijo funkcije
%pod pogoji
%Iskane spremenljivke $x_j$ so v večini paketov privzeto omejene na
%pozitivne vrednosti.
%
%Rutine za linearno programiranje so v različnih zbirkah med drugim na
%voljo pod imeni {\tt simplx} (\emph{Numerical Recipes}), {\tt
%  LinearProgramming} (\emph{Mathematica}), {\tt linprog}
%(\emph{matlab}) in v knjižnici \emph{GLPK}, ki pride tudi s
%samostojnim programom {\tt glpsol}. Ta sprejema vrsto različnih
%formatov za deklaracijo linearnih problemov (na primer {\tt CPLEX
%  LP}).
%
%}



%----------------------------------------------------------------------------------------
%	KAZALO
%----------------------------------------------------------------------------------------

\tableofcontents

%----------------------------------------------------------------------------------------
%	ZAČETEK TEKSTA
%----------------------------------------------------------------------------------------

\section{Naloga}

Želimo sestaviti najbolj optimalno sestavljen obrok, z različnimi pogoji vsebnosti hranilnih vrednosti za dieto, zdravljenje, športni obrok,... Problem rešujemo s pomočjo linearne optimizacije. Iščemo minimum funkcije (npr. maščobe):

\begin{equation}
\label{eq-mimimum-fukcija}
f(x_j)=\sum_j a_{j} x_j = \text{MIN}
\end{equation}
ob veljavnosti naslednjih pogojih.

\begin{equation*}
\begin{aligned}
\sum_j a_{ij} x_j \leq b_i\\
0 \leq x_j
\end{aligned}
\end{equation*}
V primeru, da iščemo maximum zgornje funkcije ali da imamo pri pogojih obrnmjen znak ($\geq $), enačbe oz. pogoj pomnožimo z $-1$. Na koncu je potem potrebno paziti, ali je še potrebno naše rešitve pomnožiti z $-1$.


\section{Minimizacija posamezne komponente}
Iz datoteke dostopne na {\tt http://predmeti.fmf.uni-lj.si/modelska/podatki/tabela-zivil.dat}\footnote{Dosegljiva 14.10.2015} sem uvozil tabelo, katera vsebuje energijske vrednosti, maščobe, ogljikovih hidratov, proteinov, kalcija in železa za posamezno živilo.
\\
 Najprej želimo minimizirati energijsko vrednost obroka. V tem primeru je koeficient $a_i$ energijska vrednost $j$-tega živila na $100g$, koeficient $a_{ij}$ matrike $A$ so $i$-ta vsebnost (npr.:maščobe, ogljikovi hidrati,...) $j$-tega živila. V vektorju $b$, njegove komponente $b_i$ pomenijo $i$-to omejitev. Oglejmo si preprost primer z naslednjo omejitvijo:
\begin{equation}
\label{eq-pogoj-prva}
    b =
    \begin{bmatrix}
        \text{maščobe} \\
        \text{ogljikovi hidrati} \\
        \text{proteini} \\
        \text{kalcij} \\
        \text{zelezo}
    \end{bmatrix}
    \geq
        \begin{bmatrix}
        \text{70g} \\
        \text{310g} \\
        \text{50g} \\
        \text{1000mg} \\
        \text{18mg}
    \end{bmatrix}
\end{equation}
in skupno maso manj kot $2000g$. Po vnosu podatkov v program matlab in uporabi funkcije {\tt linprog} sem dobil naslednji rezultat:

\begin{table}[h]
\centering
\label{my-label}
\begin{tabular}{lll|l|l|}
\cline{4-5}
                               &                              &  & energijska vsebnost-minimum & 1526,10 kcal \\ \cline{1-2} \cline{4-5} 
\multicolumn{1}{|l|}{pomfri}   & \multicolumn{1}{l|}{477,42g} &  & maščobe                     & 70g     \\ \cline{1-2} \cline{4-5} 
\multicolumn{1}{|l|}{bel kruh} & \multicolumn{1}{l|}{291,75g} &  & ogljikovi hidrati           & 310g    \\ \cline{1-2} \cline{4-5} 
\multicolumn{1}{|l|}{solata}   & \multicolumn{1}{l|}{1044,67} &  & proteini                    & 56,1g   \\ \cline{1-2} \cline{4-5} 
\multicolumn{1}{|l|}{grozdje}  & \multicolumn{1}{l|}{186,16g} &  & kalcij                      & 1000mg  \\ \cline{1-2} \cline{4-5} 
                               &                              &  & železo                      & 21,2mg  \\ \cline{4-5} 
\end{tabular}
\end{table}
Vsa ostala živila iz datoteke, ne bodo zastopana v našem obroku, saj so dobila vrednosti $0g$.\\
Oglejmo si še za primer, ko minimiziramo vnos maščob, ob minimalnem vnosu $2000kcal$. Skupna masa obroka je še vedsno manj kot $2000g$. Torej, sledi pogoj:
\begin{equation}
\label{eq-vec-od-1}
    b =
    \begin{bmatrix}
        \text{energijska vrednost} \\
        \text{ogljikovi hidrati} \\
        \text{proteini} \\
        \text{kalcij} \\
        \text{zelezo}
    \end{bmatrix}
    =
        \begin{bmatrix}
        \text{2000kcal} \\
        \text{310g} \\
        \text{50g} \\
        \text{1000mg} \\
        \text{18mg}
    \end{bmatrix}
    \leq
    A \vec{x}
\end{equation}
in rešitev:
\begin{table}[!h]
\centering
\begin{tabular}{lll|l|l|}
\cline{4-5}
                               &                           &  & maščobe-minimum     & 14,9g  \\ \cline{1-2} \cline{4-5} 
\multicolumn{1}{|l|}{bel kruh} & \multicolumn{1}{l|}{223g} &  & energijska vrednost & 2000kcal \\ \cline{1-2} \cline{4-5} 
\multicolumn{1}{|l|}{fižol}    & \multicolumn{1}{l|}{961g} &  & ogljikovi hidrati   & 361,1g   \\ \cline{1-2} \cline{4-5} 
\multicolumn{1}{|l|}{solata}   & \multicolumn{1}{l|}{816g} &  & proteini            & 119,7g   \\ \cline{1-2} \cline{4-5} 
                               &                           &  & kalcij              & 1000mg   \\ \cline{4-5} 
                               &                           &  & železo              & 34,5mg   \\ \cline{4-5} 
                               &                           &  & skupna masa         & 2000g    \\ \cline{4-5} 
\end{tabular}
\end{table}
\\
Če pogledamo v tabelo vseh živil, vidimo, da tri živila ne vsebujejo maščobe (rdeča pesa, pivo in grozdje). Naivno oko, bi reklo, da bodo ta živila v obroku, pa jih ni. Izbrana živila v naše obroku (kruh, fižol in solata) imajo tudi majhno vsebnost maščob, vendar boljše pripomorejo k izpolnitvi kriterija v \ref{eq-vec-od-1}.

\section{Uravnoteženost obroka}
Ker smo v naš "idealen" obrok do sedaj dobili samo tri oz. štiri živila, bom sedaj uporabili še dodaten kriterij: vsakega živila je lahko v obroku maksimalno $100g$. Ostale pogoje uporabimo iz našega prvega primera ( \ref{eq-pogoj-prva} minimizacija energijske vrednosti). Dobimo naslednjo rešitev:
\begin{table}[!h]
\centering
\begin{tabular}{|l|l|lll}
\cline{1-2} \cline{4-5}
Ovseni\_kosmici & 57,4g & \multicolumn{1}{l|}{} & \multicolumn{1}{l|}{energijska vrednost-minimum} & \multicolumn{1}{l|}{1965,8kcal} \\ \cline{1-2} \cline{4-5} 
Jabolko         & 100g  & \multicolumn{1}{l|}{} & \multicolumn{1}{l|}{maščobe}                     & \multicolumn{1}{l|}{70g}        \\ \cline{1-2} \cline{4-5} 
Pomfri          & 100g  & \multicolumn{1}{l|}{} & \multicolumn{1}{l|}{ogljikovi hidrati}           & \multicolumn{1}{l|}{310g}       \\ \cline{1-2} \cline{4-5} 
Mleko           & 100g  & \multicolumn{1}{l|}{} & \multicolumn{1}{l|}{proteini}                    & \multicolumn{1}{l|}{58g}        \\ \cline{1-2} \cline{4-5} 
Sir             & 43,2g & \multicolumn{1}{l|}{} & \multicolumn{1}{l|}{kalcij}                      & \multicolumn{1}{l|}{1000mg}     \\ \cline{1-2} \cline{4-5} 
Kruh\_bel       & 100g  & \multicolumn{1}{l|}{} & \multicolumn{1}{l|}{železo}                      & \multicolumn{1}{l|}{18mg}       \\ \cline{1-2} \cline{4-5} 
Riz             & 100g  & \multicolumn{1}{l|}{} & \multicolumn{1}{l|}{skupna masa}                 & \multicolumn{1}{l|}{1344g}      \\ \cline{1-2} \cline{4-5} 
Cokolada        & 55,0g &                       &                                                  &                                 \\ \cline{1-2}
Rdeca\_pesa     & 100g  &                       &                                                  &                                 \\ \cline{1-2}
Solata          & 100g  &                       &                                                  &                                 \\ \cline{1-2}
Zelje           & 100g  &                       &                                                  &                                 \\ \cline{1-2}
Grozdje         & 100g  &                       &                                                  &                                 \\ \cline{1-2}
Jagode          & 100g  &                       &                                                  &                                 \\ \cline{1-2}
Makaroni        & 88,9g &                       &                                                  &                                 \\ \cline{1-2}
Torta           & 100g  &                       &                                                  &                                 \\ \cline{1-2}
\end{tabular}
\end{table}
\\
Izbrana metoda za določitev optimalnega obroka ni slaba. Poleg tega, da smo popestrili obrok, smo se z minimizacijo energijske vrednosti približali vrednosti, ki naj bi bila optimalna dnevno vnešena energijska vrednost ($2000kcal$). 
\pagebreak
\section{Najcenejši obrok}
Oglejmo si kako bi izgledal obrok, če bi bil kriterij najmanjša cena. Podatke o ceni živil sem našel na spletni strani enega večjih trgovcev Sloveniji. Obrok bi izgledal sledeče:
\begin{table}[!h]
\centering
\begin{tabular}{lll|l|l|}
\cline{1-2} \cline{4-5}
\multicolumn{1}{|l|}{Mleko}     & \multicolumn{1}{l|}{61g}  &  & cena-najceneje      & 1,3\euro      \\ \cline{1-2} \cline{4-5} 
\multicolumn{1}{|l|}{Kruh\_bel} & \multicolumn{1}{l|}{647g} &  & mascobe             & 70g        \\ \cline{1-2} \cline{4-5} 
\multicolumn{1}{|l|}{Maslo}     & \multicolumn{1}{l|}{55g}  &  & energijska vrednost & 2149,8kcal \\ \cline{1-2} \cline{4-5} 
                                &                           &  & ogljikovi hidrati   & 310g       \\ \cline{4-5} 
                                &                           &  & proteini            & 73,0g      \\ \cline{4-5} 
                                &                           &  & kalcij              & 1000mg     \\ \cline{4-5} 
                                &                           &  & železo              & 22,4mg     \\ \cline{4-5} 
                                &                           &  & skupna masa         & 762,2g     \\ \cline{4-5} 
\end{tabular}
\end{table}
\\
Ponovno imamo samo tri različne živila na razpolago, zato ponovno omejimo vsa živila na maksimalno vrednost $100g$ (npr. zaradi finančnih razlogov, si ne moremo privoščiti več). Pridemo do obroka naslednjih vrednosti:
\begin{table}[!h]
\centering
\begin{tabular}{|l|l|}
\hline
cena-najceneje      & 2,73\euro \\ \hline
mascobe             & 70g       \\ \hline
energijska vrednost & 2037g     \\ \hline
ogljikovi hidrati   & 310g      \\ \hline
proteini            & 74g       \\ \hline
kalcij              & 1000mg    \\ \hline
železo              & 18mg      \\ \hline
skupna masa         & 762,2g    \\ \hline
\end{tabular}
\end{table}
\\
Sedaj se v obroku ne pojavijo sledeča živila :
\begin{itemize}
\item živalskega izvora: govedina, svinjina, skuša, jajca
\item zeljenjava: fižol, solata, jagoda
\item čokolada, torta
\end{itemize}

\pagebreak
\end{document}
